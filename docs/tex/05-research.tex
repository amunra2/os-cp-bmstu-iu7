\chapter{Исследовательская часть}

\section{Условия исследования}

Исследование проводилось на компьютере со следующими характеристиками:

\begin{itemize}
    \item операционная система: Ubuntu 22.04 \cite{ubuntu} Linux \cite{linux} x86\_64;
    \item процессор Intel(R) Core(TM) i5-7300HQ @ 2.50GHz;
    \item память ОЗУ 8 GB.
\end{itemize}


\section{Результат исследования}

Исследование заключается в сравнении двух простых программ на языке \texttt{C++}. Различие заключается в том, что одна программа является консольным приложением, а вторая программа имеет графический интерфейс.

На рисунке \ref{img:research-console} приведена информация о процессе, который был запущен консольным приложением, а на рисунке \ref{img:research-gui} --- приложением с графическим интерфейсом.

\imgs{research-console}{h!}{0.4}{Информация о процессе консольного приложения}
\imgs{research-gui}{h!}{0.4}{Информация о процессе с графическим интерфейсом}


\section*{Вывод}

Как видно из полученных результатов, консольное приложение для своей работы использует только 3 файла, в то время как приложение с графическим интерфейсом использует большое количество файлов. Также из данных по памяти видно, что приложение с графическим интерфейсом занимает больше оперативной памяти, чем консольное приложение. Стоит отметить, что подобное сравнение было проведено благодаря разработанному в рамках курсовой работы программному продукту.

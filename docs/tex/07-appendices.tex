\addcontentsline{toc}{chapter}{ПРИЛОЖЕНИЕ А}
\chapter*{ПРИЛОЖЕНИЕ А} \label{appendices:a}

\mylisting[c]{task-struct.c}
        {firstline=1,lastline=23}{Структура \texttt{task\_struct}}{task-struct}{}


\addcontentsline{toc}{chapter}{ПРИЛОЖЕНИЕ Б} 
\chapter*{ПРИЛОЖЕНИЕ Б} \label{appendices:b}

\mylisting[c]{vm-area-struct.c}
        {firstline=1,lastline=14}{Структура \texttt{vm\_area\_struct}}{vm-area-struct}{}


\addcontentsline{toc}{chapter}{ПРИЛОЖЕНИЕ В} 
\chapter*{ПРИЛОЖЕНИЕ В} \label{appendices:c}

\mylisting[c]{fdtable.c}
        {firstline=1,lastline=9}{Структура \texttt{fdtable}}{fdtable}{}


\addcontentsline{toc}{chapter}{ПРИЛОЖЕНИЕ Г} 
\chapter*{ПРИЛОЖЕНИЕ Г} \label{appendices:d}

\mylisting[c]{proc-dir-entry.c}
        {firstline=1,lastline=15}{Структура \texttt{proc\_dir\_entry}}{proc-dir-entry}{}       


\addcontentsline{toc}{chapter}{ПРИЛОЖЕНИЕ Д} 
\chapter*{ПРИЛОЖЕНИЕ Д} \label{appendices:e}

\mylisting[c]{main-kernel.c}
        {firstline=57,lastline=105}{Функции записи и чтения для структуры \texttt{file\_operations} получения памяти процесса}{main-read-write}{}


\addcontentsline{toc}{chapter}{ПРИЛОЖЕНИЕ E} 
\chapter*{ПРИЛОЖЕНИЕ E} \label{appendices:f}

\mylisting[c]{func-kernel.c}
        {firstline=1,lastline=92}{Функция получения информации о памяти}{mem-func}{}

\mylisting[c]{func-kernel.c}
        {firstline=95,lastline=169}{Функция получения информации об открытых процессом файлах}{files-func}{}

\mylisting[c]{func-kernel.c}
        {firstline=171,lastline=243}{Функции получения дерева процессов в системе}{tree-func}{}
